MLP 的实现总结

吴恩达-深度学习第一章作业,我已经完成了,python 神经网络那本书也已经看了几遍,所以理论知识自认为已经不错了.
不过在实现的过程中仍然遇到了很多问题.这也是要自己实现一遍的意义.

给自己定的目标: 实现不同于上面两门课的实现, 实现加入自己的理解.
思考的问题:
1.如何鉴别自己的实现是对的,更近一步说,如何写测试用例?
2.如何调试. 

关于测试用例.我沿用了吴达恩里面的用例(这比较取巧)把我的实现作为一个黑盒,验证比较简单.
不过我也考虑了一下比较恶劣的环境.比如我最近在根据一篇论文实现一个网络架构.除了论文,什么代码都没有.
我如何验证我的代码对不对呢?比如我会根据论文中提到的相似架构, 查找对应的论文和源码(如果幸运的话),
通过已有代码,组装论文中的代码. 根据论文中描述的过程,处理数据,尽量模拟相同的数据分布. 按照论文的超参数进行设置.
看看训练出的结果是否相同.如果不相同,目前我能想到的方法就是联系作者了.

关于调试. 我用我的实现,其网络结构和练习中的结构一样.然后比较训练参数.来定位问题.依旧很取巧.
现实环境会非常恶劣.这个时候,数据可视化,打印关键数据(比如损失变化)就很重要了.比如我开始训练的时候,发现损失一直增大,
然后打印了 每一层的矩阵.发现很多inf. 而 inf 是因为传入sigmoid的值很大, 最终定位到是没有乘学习率.导致权重增长很快.

beg

